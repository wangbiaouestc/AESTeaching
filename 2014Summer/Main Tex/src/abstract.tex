Video en- and decoding is nowadays a central technology on many platforms and systems in all over the world. It is used in computers, TVs, smartphones, streaming platforms, video conferences, blu-rays etc.
It has been developed over the past decades and therefore many different video codecs and algorithms were developed over time to solve certain purposes (e.g., H.265/HEVC, H.264/AVC, MPEG-2) and to address requirements of many applications.\\
Since the demand of high resolution videos is more and more increasing these codecs aim to reach a high compression rate while maintaining a good quality. The cost of this approach is always a higher computing requirement 
(e.g. HEVC aims to reach half the size while maintaining the same quality as AVC \todo{this example does not show the higher computation requirement for HEVC, remove it}).\\
In order to meet these requirements many different basic approaches are researched. This articles mainly \todo{typo focusses} on the graphic processing units (GPU) and the encoding process.\\
GPUs have emerged as a coprocessing unit for central processing units (CPUs) to accelerate certain logical parts of the encoding process.\\
GPUs consist of a large amount of streaming processors, which allow them to execute a lot of operations in parallel. Because GPUs consist of a \todo{capitalize Single Instruction Multiple Data} (SIMD) architecture  and the fact that data synchronisation between CPU and GPU is considered a bottleneck, it is a rather complex task to efficiently use this immense parallel performance while handling all the constraints of the GPUs .\\
In this paper three different strategies to address this challenge are reviewed. Two of them propose parallel strategies for the H.264 codec and one for the H.265 codec. They all reach certain speed ups in their field but are not necessarily developed for wide use.